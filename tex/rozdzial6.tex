	\newpage
\section{Scenariusze przypadków użycia}  %6
\textbf{a) Scenariusze użytkownika.}
\newline \textit{\underline{Scenariusz nr 1}}
\newline \textbf{Tytuł:} Logowanie
\newline \textbf{Warunek wejścia:} Posiada konto
\newline \textbf{Przebieg:} Użytkownik podejmuje próbę logowania. Podaje niezbędne dane (login, hasło).  
\newline \textbf{Zakończenie:} Jeśli próba logowania jest pozytywna - zyskuje dostęp do pełnej funkcjonalności aplikacji, w przeciwnym razie dostaje kolejną możliwość wprowadzenia danych logowania.
\newline \textbf{Zakończenie alternatywne:} Jeśli użytkownik nie posiada konta dostanie możliwość wykonania scenariusza nr 2 (Rejestracja).
\newline\newline \textit{\underline{Scenariusz nr 2}}, 
\newline \textbf{Tytuł:} Rejestracja
\newline \textbf{Warunek wejścia:} Musi wejść do aplikacji (kliknąć w ikonę aplikacji).
\newline \textbf{Przebieg:} Użytkownik dostaje możliwość stworzenia swojego profilu, które jest konieczne do skorzystania z aplikacji.
\newline \textbf{Zakończenie:} Jeśli rejestracja przebiegnie pomyślnie użytkownik dostanie możliwość zalogowania się do aplikacji.
\newline \textbf{Zakończenie alternatywne:} W przypadku podania błędnych danych nastąpi wyświetlenie komunikatu błędu i możliwość podjęcia kolejnej próby.
\newline\newline \textit{\underline{Scenariusz nr 3}}
\newline \textbf{Tytuł:} Edycja profilu
\newline \textbf{Warunek wejścia:} Użytkownik jest zalogowany.
\newline \textbf{Przebieg:} Użytkownik dostaje możliwość edycji profilu tzn. m. in. zmiany swojego hasła.
\newline \textbf{Zakończenie:} Po poprawnej edycji danych następuje ich zmiana.
\newline\newline \textit{\underline{Scenariusz nr 4}}
\newline \textbf{Tytuł:} Sprawdzenie osiągnięć
\newline \textbf{Warunek wejścia:} Użytkownik jest zalogowany i znajduje się w głównym menu.
\newline \textbf{Przebieg:} Użytkownik w menu głównym klika w Osiągnięcia.
\newline \textbf{Zakończenie:} Po kliknięciu następuje przejście do widoku osiągnięć, gdzie użytkownik ma możliwość sprawdzenia swoich statystyk.
\newline\newline \textit{\underline{Scenariusz nr 5}}
\newline \textbf{Tytuł:} Wybór kategorii pytań
\newline \textbf{Warunek wejścia:} Użytkownik jest zalogowany i znajduje się w głównym menu.
\newline \textbf{Przebieg:} Użytkownik w menu głównym klika w Kategorie pytań.
\newline \textbf{Zakończenie:} Po kliknięciu następuje przejście do widoku kategorii pytań z danej dziedziny wiedzy, gdzie użytkownik wybiera kategorie z jakiej chce odpowiadać.
\newline \textbf{Zakończenie alternatywne:} Użytkownik chcąc wyjść z danej kategorii klika przycisk wstecz i wychodzi do głównego menu. 


\textbf{b) Scenariusze administratora.}
\newline \textit{\underline{Scenariusz nr 1}}
\newline \textbf{Tytuł:} Logowanie
\newline \textbf{Warunek wejścia:} brak
\newline \textbf{Przebieg:} Administrator loguje się do systemu, dzięki czemu zyskuje dostęp do pełnej funkcjonalności i funkcji administratorskich.
\newline \textbf{Zakończenie:} Zalogowanie do konta oraz możliwość zarządzania poszczególnymi częściami serwisu.
