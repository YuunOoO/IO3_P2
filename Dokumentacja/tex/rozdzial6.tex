	\newpage
\section{Scenariusze przypadków użycia}  %6
\textbf{a) Scenariusze użytkownika.}
\newline \textit{\underline{Scenariusz nr 1}}
\newline \textbf{Tytuł:} Logowanie
\newline \textbf{Warunek wejścia:} Posiada konto
\newline \textbf{Przebieg:} Użytkownik podejmuje próbę logowania. Podaje niezbędne dane (login, hasło).  
\newline \textbf{Zakończenie:} Jeśli próba logowania jest pozytywna - zyskuje dostęp do pełnej funkcjonalności aplikacji, w przeciwnym razie dostaje kolejną możliwość wprowadzenia danych logowania.
\newline \textbf{Zakończenie alternatywne:} Jeśli użytkownik nie posiada konta dostanie możliwość wykonania scenariusza nr 2 (Rejestracja).
\newline\newline \textit{\underline{Scenariusz nr 2}}, 
\newline \textbf{Tytuł:} Rejestracja
\newline \textbf{Warunek wejścia:} Musi wejść do aplikacji (kliknąć w ikonę aplikacji).
\newline \textbf{Przebieg:} Użytkownik dostaje możliwość stworzenia swojego profilu, który jest konieczny do skorzystania z aplikacji.
\newline \textbf{Zakończenie:} Jeśli rejestracja przebiegnie pomyślnie użytkownik dostanie możliwość zalogowania się do aplikacji.
\newline \textbf{Zakończenie alternatywne:} W przypadku podania błędnych danych nastąpi wyświetlenie komunikatu błędu i możliwość podjęcia kolejnej próby.
\newline\newline \textit{\underline{Scenariusz nr 3}}
\newline \textbf{Tytuł:} Sprawdzenie osiągnięć
\newline \textbf{Warunek wejścia:} Użytkownik jest zalogowany i znajduje się w głównym menu.
\newline \textbf{Przebieg:} Użytkownik w menu głównym klika w Osiągnięcia.
\newline \textbf{Zakończenie:} Po kliknięciu następuje przejście do widoku osiągnięć, gdzie użytkownik ma możliwość sprawdzenia swoich statystyk z odbytych gier.
\newline\newline \textit{\underline{Scenariusz nr 4}}
\newline \textbf{Tytuł:} Wybór kategorii pytań
\newline \textbf{Warunek wejścia:} Użytkownik jest zalogowany i znajduje się w głównym menu.
\newline \textbf{Przebieg:} Użytkownik w menu głównym klika w Kategorie pytań.
\newline \textbf{Zakończenie:} Po kliknięciu następuje przejście do widoku kategorii pytań z danej dziedziny wiedzy, gdzie użytkownik wybiera kategorie z jakiej chce odpowiadać.
\newline\newline \textit{\underline{Scenariusz nr 5}}
\newline \textbf{Tytuł:} Gra - odpowiadanie na pytania
\newline \textbf{Warunek wejścia:} Użytkownik wybrał kategorie z jakiej chce odpowiadać.
\newline \textbf{Przebieg:} Po wybraniu interesującej kategorii, rozpoczyna się gra, w której użytkownikowi ukazuje się losowe pytanie z czterema odpowiedziami do wyboru.   
\newline \textbf{Zakończenie:} Użytkownik wybiera jego zdaniem poprawną odpowiedź, a następnie następuje weryfikacja zaznaczonej odpowiedzi.
\newline \textbf{Zakończenie alternatywne:} Może zdarzyć się, że użytkownik nie zaznaczy żadnej z podanych odpowiedzi, wtedy po upływie ustalonego czasu nastąpi weryfikacja odpowiedzi. 
\newline\newline \textit{\underline{Scenariusz nr 6}}
\newline \textbf{Tytuł:} Weryfikacja odpowiedzi
\newline \textbf{Warunek wejścia:} Użytkownik zaznaczył odpowiedź, bądź nie.
\newline \textbf{Przebieg:} Po wybraniu odpowiedzi, czy też upływie czasu - automatycznie przez system zaznaczana jest prawidłowa odpowiedź.
\newline \textbf{Zakończenie:} Po weryfikacji generowane jest kolejne pytanie z puli. 
\newline\newline \textit{\underline{Scenariusz nr 7}}
\newline \textbf{Tytuł:} Podsumowanie wyniku i zakończenie gry.
\newline \textbf{Warunek wejścia:} Wyczerpanie się puli pytań
\newline \textbf{Przebieg:} Po przejściu 10 pytań z danej kategorii oraz weryfikacji ich poprawności następuje koniec gry, po której użytkownikowi wyświetlają się statystyki za rozgrywkę.
\newline \textbf{Zakończenie:} Po kliknięciu odpowiedniego przycisku następuje przeniesienie do głównego menu, a zdobyte punkty aktualizują się w osiągnięciach profilu.  
\newline\newline \textit{\underline{Scenariusz nr 8}}
\newline \textbf{Tytuł:} Przerwanie gry.
\newline \textbf{Warunek wejścia:} Użytkownik znajduje się w trybie gry.
\newline \textbf{Przebieg:} Użytkownik podczas rozgrywki ma możliwość przerwania jej w każdej chwili poprzez kliknięcie odpowiedniego przycisku. Gdy to zrobi, gra automatycznie się zakończy, a on znajdzie się w głównym menu. 
\newline \textbf{Zakończenie:} Po kliknięciu odpowiedniego przycisku następuje przeniesienie do głównego menu, a zdobyte punkty nie ulegają zapisaniu. 
\newline\newline \textit{\underline{Scenariusz nr 9}}
\newline \textbf{Tytuł:} Wyjście z aplikacji.
\newline \textbf{Warunek wejścia:} Użytkownik znajduje się w menu głównym.
\newline \textbf{Przebieg:} Użytkownik będąc w głównym menu klika przycisk "WYJDŹ"
\newline \textbf{Zakończenie:} Po kliknięciu następuje zamknięcie aplikacji.
\newline\newline \textbf{b) Scenariusze administratora.}
\newline \textit{\underline{Scenariusz nr 1}}
\newline \textbf{Tytuł:} Logowanie
\newline \textbf{Warunek wejścia:} Konto posiada uprawnienia administratora.
\newline \textbf{Przebieg:} Administrator loguje się do systemu, dzięki czemu zyskuje dostęp do funkcji administratorskich.
\newline \textbf{Zakończenie:} Zalogowanie do konta oraz możliwość zarządzania (m. in. podgląd listy użytkowników).

